%% LyX 2.3.6 created this file.  For more info, see http://www.lyx.org/.
%% Do not edit unless you really know what you are doing.
\documentclass[english]{article}
\usepackage[T1]{fontenc}
\usepackage[latin9]{inputenc}

\makeatletter
%%%%%%%%%%%%%%%%%%%%%%%%%%%%%% User specified LaTeX commands.
\usepackage{babel}











\makeatother

\usepackage{babel}
\begin{document}
\title{\texttt{\textbf{Simple 3D Viewer Manual}}}
\author{by JakeTheSillySnake (@ginzburg\_jake on Telegram)}
\date{Last updated December 2024}

\maketitle
\tableofcontents{}

\section{Introduction}

This document describes how to install, run, check and remove Simple
Viewer on Linux and Unix-like systems. Please note that the program
was written and tested on Ubuntu 22.04 LTS, so its behaviour may differ
if you use other versions or OS distributions.

\section{Installation \& Running}

\subsection{Prerequisites}

Correct compilation and running of the program depends on other utilities
and libraries. Check that you have their latest versions before proceeding: 
\begin{description}
\item [{App~Compilation~\&~Running}] gcc, make, Qt5 
\item [{Testing}] GTest library 
\item [{GCOV~Report}] gcov, lcov 
\item [{Leaks~Check}] valgrind 
\item [{Convert~Manual~to~DVI}] texi2dvi 
\end{description}

\subsection{Setup}

Download or clone (\texttt{git clone <link\_to\_repo>}) the source
repository to where you can easily find it. Then type and run the
following commands in the terminal: 
\begin{enumerate}
\item \texttt{cd <path-to-git-folder>/src/viewer} 
\item \texttt{make install}
\end{enumerate}
Now the program is compiled, placing the app in a folder named \texttt{SimpleViewer/}.
The app should open automatically. If you want to open it later using
command line, run\texttt{ make viewer.} If there are errors, you're
likely missing some packages. Check \textbf{Prerequisites}.

\subsection{Usage}

The app layout should be fairly intuitive. On the first load, you'll
see a horizontal \textbf{Menu Tab} plus an \textbf{Information Tab}
on the right side of the screen. The \textbf{Menu Tab} is divided
into three sub-categories and the options you choose are saved after
each session:
\begin{enumerate}
\item \textbf{File}
\begin{enumerate}
\item Upload~file --- \emph{choose .obj file to display}
\item Save~as~.bmp,~.jpg --- \emph{save current render as a picture}
\item Make~screencast --- \emph{save current render as a GIF (5sec, 10
frames/sec, 640x480) with custom model rotation}
\item Exit --- \emph{close the app}
\end{enumerate}
\item \textbf{Model}
\begin{enumerate}
\item Rotate --- \emph{rotate the model on a given axis by N degrees}
\item Translate --- \emph{translate the model on a given axis by N degrees}
\item Scale --- \emph{scale the model on a given axis by N degrees}
\item Wireframe --- \emph{display model as wireframe (as opposed to solid)}
\item Parallel projection --- \emph{apply parallel projection (as opposed
to central, or perspective projection)}
\end{enumerate}
\item \textbf{Settings}
\begin{enumerate}
\item Set background --- \emph{change background colour}
\item Set texture --- \emph{choose texture (only in SOLID MODE)}
\item Vertices --- \emph{set colour, size and display mode (circle, square,
none) of vertices (only in WIREFRAME MODE)}
\item Edges --- \emph{set colour, size and dashed mode of edges (only in
WIREFRAME MODE)}
\item Show information --- \emph{toggle the visibility of the} \textbf{Information
Tab}
\end{enumerate}
\end{enumerate}
The model is initially displayed in the central part of the screen\footnote{Tip: if you can't see your model at first, try changing its dimensions
and coordinates. Also check the settings to ensure that the model
and background are differently coloured.}. The \textbf{Information Tab} lists the keys to manipulate your model
as well as its number of edges and vertices. Additionally, you can\emph{
}\textbf{click and drag the mouse} to \textbf{rotate} your object
and \textbf{scroll with the mouse wheel }to \textbf{scale} it.

\section{Structure \& testing}

The program was made using C++17 language and standard libraries,
with the interface provided by Qt5. It uses an MVC design structure
to separate the interface from the controller and model. The source
code can be found in \texttt{viewer/backend} and \texttt{viewer/frontend}
folders. The backend libraries can be tested with GTest: 
\begin{enumerate}
\item To run tests:\texttt{ make test} 
\item To display test coverage:\texttt{ make gcov\_report}
\item To check for leaks: \texttt{make valgrind } 
\end{enumerate}
Running \texttt{make} or \texttt{make all} will reinstall and compile
the program. You can get DVI documentation with \texttt{make dvi}
or a distribution .tar.gz package with \texttt{make dist}. Also note:
\begin{enumerate}
\item Not every model can be processed by this simple app --- check\texttt{
assets/models} for inspiration
\item The basic textures and shaders in \texttt{assets/textures} and \texttt{assets/shaders}
are needed for the app to function (but you can mod and replace them
if you know what you're doing)
\item This app uses external libraries to process .gif, .jpg, .bmp and .obj
files: please see \texttt{s}rc/\texttt{external} for licenses
\end{enumerate}

\section{Deinstallation}

Simply run \texttt{make uninstall.} This will remove the \texttt{SimpleViewer/}
directory but not the original download, which can be safely deleted
afterwards.

\subparagraph{If you wish to suggest an improvement or report a bug, contact me
@ginzburg\_jake (Telegram) or JakeTheSillySnake (GitHub).}
\end{document}
